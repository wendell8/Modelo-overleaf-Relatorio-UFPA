\section{Conclusões}
Neste relatório, foi possível analisar diversas imagens, com o intuito de relatar as mudanças nos aspectos e parâmetros que as geraram. Portanto, pode-se tomar como conclusões que: o dado vai ajustar quando as densidades do modelo e a densidade calculada, que é a densidade hiperbólica, são iguais ou próximas uma da outra. O contrário dificulta o ajuste das curvas quando estiverem distante uma da outra. Do pronto de vista geológico, isso pode está relacionado à presença de materiais com uma densidade muito maior do que o meio em que ele está, como diques e soleiras. Outro fato relevante para a análise deste relatório, está relacionado com o fator de variação do contraste de densidade com a profundidade, o $\beta$. O mesmo implica que, quão maior o valor de $\rho_0$ maior será o valor de $\beta$ para que haja um bom ajuste entre as curvas, Tabela 1.

\vspace{15cm}

\section{Objetivos}

    \subsection{Objetivo geral}
        O objetivo geral deste estudo é avaliar a densidade geológica em um determinado ambiente por meio da técnica de densidade hiperbólica, realizando testes com diferentes valores de contraste de
        densidade na superfície, $\rho_0$ e fator de variação do contraste de densidade com a profundidade $\beta$. As análises serão baseadas nas curvas ajustadas e distantes da Figura \ref{Figura 1}, bem como nas imagens obtidas durante os testes para justificar as interpretações dos resultados.
            \subsection{Objetivos específicos}
        \begin{itemize}
            \item Utilizar a técnica de densidade hiperbólica para estimar a densidade de rochas e minerais em subsuperfície 
            \item Variar os parâmetros do contraste de densidade na superfície e do fator de variação da densidade com a profundidade para encontrar valores que produzam anomalias gravimétricas semelhantes às geradas pela densidade hiperbólica.
            \item Identificar as condições que os ambientes geológicos apresentam (relacionadas às propriedades físicas das rochas) para produzir o comportamento dos testes 1 (curvas ajustadas Figura 1a) e testes 2 (curvas distantes Figura 1a).
            \item Explicar as razões físicas do porque ocorrem curvas que não estão ajustadas, já que a fonte gravimétrica é a mesma.
        \end{itemize}